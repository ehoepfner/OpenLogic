% Part: first-order-logic
% Chapter: sequent-calculus
% Section: soundness.tex

\documentclass[../../../include/open-logic-section]{subfiles}

\begin{document}

\olfileid{fol}{seq}{sou}
\olsection{Soundness}

\begin{explain}
A !!{derivation} system, such as the sequent calculus, is \emph{sound}
if it cannot !!{derive} things that do not actually hold.  Soundness is
thus a kind of guaranteed safety property for !!{derivation} systems.
Depending on which proof theoretic property is in question, we would
like to know for instance, that
\begin{enumerate}
\item every !!{derivable} !!{sentence} is valid;
\item if a !!{sentence} is !!{derivable} from some others, it is also a
  consequence of them;
\item if a set of !!{sentence}s is inconsistent, it is unsatisfiable.
\end{enumerate}
These are important properties of a !!{derivation} system.  If any of them do
not hold, the !!{derivation} system is deficient---it would !!{derive} too much.
Consequently, establishing the soundness of a !!{derivation} system is of the
utmost importance.

Because all these proof-theoretic properties are
defined via !!{derivability} in the sequent calculus of certain sequents,
proving (1)--(3) above requires proving something about the semantic
properties of !!{derivable} sequents.  We will first define what it means
for a sequent to be \emph{valid}, and then show that every !!{derivable}
sequent is valid.  (1)--(3) then follow as corollaries from this
result.
\end{explain}

\begin{defn}
A !!{structure}~$\Struct M$ \emph{satisfies} a sequent $\Gamma
\Sequent \Delta$ iff for all variable assignments $s$ either $\Sat/{M}{!E}[s]$
for some $!E \in \Gamma$ or $\Sat{M}{!E}[s]$ for some $!E \in \Delta$.

A sequent is \emph{valid} iff every !!{structure}~$\Struct M$
satisfies it.
\end{defn}

\begin{thm}[Soundness]
\ollabel{sequent-soundness} If $\Log{LK}$ !!{derive}s $\Gamma \Sequent
\Delta$, then $\Gamma \Sequent \Delta$ is valid.
\end{thm}

\begin{proof}
Let $\Pi$ be a !!{derivation} of $\Gamma \Sequent \Delta$. We proceed by
induction on the height of the proof tree $\Pi$.

If the height is~1, then $\Pi$ consists only of an
initial sequent. Every initial sequent $!A \Sequent !A$ is obviously
valid, since for every $\Struct M$, either $\Sat/{M}{!A}[s]$ or
$\Sat{M}{!A}[s]$, as well as the initial sequents $\quad \Sequent \ltrue$
and $\lfalse \Sequent \quad$.

If the height $n$ is greater than~$1$, we distinguish cases
according to the type of the lowermost inference. By induction
hypothesis, we can assume that the premises of that inference are
valid, since the height of the proof of a premise is smaller than $n$.

Hence we have to show for every rule that, if all upper sequents are valid, 
then the lower sequent is valid. Then we are done.
\begin{enumerate}
\item[\LeftR{\Weakening}:] If $\Gamma \Sequent \Delta$ is valid, then for all 
$\Struct M$, $s$ we have either $\Sat/{M}{!E}[s]$ some $!E \in \Gamma$  or $
\Sat{M}{!E}[s]$ some $!E \in \Delta$. Since $\Gamma \subseteq \{!A\} \cup  
\Gamma$, $!A, \Gamma \Sequent \Delta$ is valid as well.
\item[\RightR{\Weakening}:] Exercise.
\item[\LeftR{\lnot}:] If $\Gamma \Sequent \Delta, !A$ is valid, then for all $
  \Struct M$, $s$ we have either $\Sat/{M}{!E}[s]$ for some $!E \in \Gamma$  
or $ \Sat{M}{!E}[s]$ for some $!E \in \Delta \cup \{!A\}$. In the latter 
case, we have $\Sat{M}{!E}[s]$ for some $ !E \in \Delta$ or $\Sat/{M}{\lnot !A
}[s]$. Hence $\lnot !A, \Gamma \Sequent \Delta$ is valid. 
\item[\RightR{\lnot}:] Exercise.
\item[\LeftR{\land}:] If $!A, \Gamma \Sequent \Delta$ is valid, then for all $
  \Struct M$, $s$  we have either $\Sat/{M}{!E}[s]$ for some $!E \in \{!A\} 
\cup \Gamma$  or $\Sat{M}{!E}[s]$ for some $!E \in \Delta$. In the first 
case, we have $\Sat/{M }{!E}[s]$ for some $!E \in \Gamma$ or $\Sat/{M}{!A}[s]$
, thus $\Sat/{M}{!A \land !B }[s]$ and $\Sat/{M}{!B \land !A}[s]$ as well. 
Consequently, $!A \land !B, \Gamma \Sequent \Delta$ and $!B \land !A, \Gamma 
\Sequent \Delta$ are valid.
\item[\RightR{\lor}:] Exercise.
\item[\RightR{\lif}:] If $!A, \Gamma \Sequent \Delta, !B$ is valid, then for 
all $   \Struct M$, $s$  we have either $\Sat/{M}{!E}[s]$ for some $!E \in \{ 
!A\} \cup \Gamma$  or $\Sat{M}{!E}[s]$ for some $!E \in \Delta \cup \{!B\}$. 
In the first case, we have  $\Sat/{M}{!E}[s]$ for some $!E \in \Gamma$ or $ 
\Sat/{M}{!A}[s]$, thus $\Sat{M}{!A \lif !B}[s]$. In the latter case, we have $
 \Sat{M}{!E}[s]$ for some $!E \in \Delta$ or $\Sat{M}{!B}[s]$, thus $\Sat{M}{
!A \lif !B}[s]$. Hence $\Gamma \Sequent \Delta, !A \lif !B$ is valid.
\item[\LeftR{\lforall}:] If $\Subst{!A}{t}{x}, \Gamma \Sequent \Delta$ is 
valid, then for all $\Struct M$, $s$ we have either $\Sat/{M}{!E}[s]$ for 
some $!E \in \{ \Subst{!A}{t}{x}\} \cup \Gamma$  or $\Sat{M}{!E}[s]$ for some 
$!E \in \Delta$. In the first case, we have $\Sat/{M}{!E}[s]$ for some $!E \in
 \Gamma$ or $\Sat/{M}{\Subst{!A}{t}{x}}[s]$, thus $\Sat/{M}{\lforall x !A}[s] 
$, because for the $x$-variant $s'$ of $s$ given by $s'(x) = \Value{t}{M}[s]$ 
we have $ \Sat{M}{!A}[s']$ iff $\Sat{M}{\Subst{!A}{t}{x}}[s]$. Hence $\lforall
 x !A, \Gamma \Sequent \Delta$ is valid.
\item[\RightR{\lexists}:] Exercise.
\item[\LeftR{\lexists}:] Assume $\Subst{!A}{y}{x}, \Gamma \Sequent \Delta$ is 
valid. Let $\Struct M$, $s$ with $\Sat{M}{\{\lexists x !A\} \cup \Gamma}[s] $ 
be arbitrary. In particular, $\Sat{M}{\lexists x !A}[s]$.  Hence there is a $ 
x $-variant $s'$ of $ s$ such that $\Sat{M}{!A}[s']$. 
If $x \equiv y$, then $\Subst{!A}{y}{x} \equiv !A$, consequently $\Sat{M}{!A}[
  s']$ iff $\Sat{M}{\Subst{!A}{y}{x}}[s']$. Since $x$ is free in $\Gamma$ and 
$ \Delta$, we have $\Sat{M}{ !E}[s]$ iff $\Sat{M}{!E}[s']$ for all $!E \in 
\Gamma \cup \Delta$. Now the premise gives us either $  \Sat/{M}{!E}[s']$ for 
some $ !E \in \Gamma$ or $\Sat{M}{!E}[s']$ for some $ !E \in \Delta$. Thus, 
there is some $!E \in \Delta$ such that $\Sat{M}{!E}[s]$.
Otherwise, $x \not\equiv y$. Because $y$ does not occur free in $\lforall x !A
$  , it holds $!A \equiv \Subst{\Subst{!A}{y}{x}}{x }{y }$. By 
\olref[fol][syn][ext]{prop:ext-formulas} 
we have $\Sat{M}{\Subst{ \Subst{!A}{y }{x}}{x}{y }}
[ s' ]$ iff $\Sat{ M}{\Subst{!A}{y}{x}}[s'']$ for the $y$-variant $s''$ of $ 
s'$  given by $s''(y) = s'(x)$. Since $x$ is free in $\Subst{!A}{y}{x }$, we 
get $  \Sat{M}{\Subst{!A}{y}{x}}[s'']$ iff $\Sat{M }{\Subst{!A}{y}{x}}[ s''']$
 for the $x$-variant $s'''$ of $s''$ given by $s''' (x) = s(x)$. Notice that $
s'''$   is also a $y$-variant of $s$. Since $y$ is free in $\Gamma$ and $
\Delta$, we have $\Sat{M}{ !E}[s]$ iff $\Sat{M}{!E}[s''']  $ for all $!E \in 
\Gamma \cup \Delta$. Because the premise, we have either $ \Sat/{M}{!E}[s''']$
 for some $ !E \in \Gamma$ or $\Sat{M}{!E}[s''']$ for some $!E \in \Delta$. 
Consequently, there is some $!E \in \Delta$ such that $\Sat{M}{!E}[s]$.
Hence $\lexists x !A, \Gamma \Sequent \Delta$ is valid.
\item[\RightR{\lforall}:] Exercise.
\item[cut:] Suppose $\Gamma \Sequent \Delta, !A$ and $!A, \Pi \Sequent \Lambda 
$ are valid. Let $\Struct M$, $s$ be arbitrary. If $\Sat{M}{!A}[s]$, then by 
the right premise we have either $\Sat/{M}{!E}[s]$ for some $!E \in \Pi$ or $ 
\Sat{M}{!E}[s]$ for some $!E \in \Lambda$. Otherwise, if $\Sat/{M}{!A}[s]$, 
then by the left premise we have either $\Sat/{M}{!E}[s]$ for some $!E \in 
\Gamma$ or $\Sat{M }{!E}[s]$ for some $!E \in \Delta$.  Hence $\Gamma, \Pi 
\Sequent \Delta, \Lambda$ is valid as well.
\item[\RightR{\land}:] Suppose $\Gamma \Sequent \Delta, !A$ and $\Gamma 
\Sequent \Delta, !B $ are valid. Let $\Struct M$, $s$ be arbitrary. If $\Sat/{
M}{!A \land !B}[s]$, then either $\Sat/{M}{!A}[s]$ or $\Sat/{M}{!B}[s]$. In 
either case we conclude $\Sat/{M}{!E}[s]$ for some $!E \in \Gamma$ or $\Sat{M
}{!E}[s]$ for some $!E \in \Delta$. Consequently, $\Gamma \Sequent \Delta, !A 
\land !B$ is valid.
\item[\LeftR{\lor}:] Exercise.
\item[\LeftR{\lif}:] Suppose $\Gamma \Sequent \Delta, !A$ and $!B, \Pi 
\Sequent \Lambda $ are valid. Let $\Struct M$, $s$ be arbitrary. If $\Sat{M}{
!A \lif !B}[s]$, then $\Sat/{M}{!A}[s]$ or $\Sat{M}{!B}[s]$. In the first 
case, have we conclude either $\Sat/{M}{!E}[s]$ for some $ !E \in \Gamma$ or $
\Sat{M}{!E}[s]$ for some $!E \in \Delta$. In the other case, we have either $
\Sat/{M}{!E}[s]$ for some $!E \in \Pi$ or $\Sat{M}{!E}[s]$ for some $!E \in 
\Lambda$. Thus $!A \lif !B, \Gamma, \Pi \Sequent \Delta, \Lambda$ is valid as 
well.
\end{enumerate}
\end{proof}

\begin{prob}
Complete the proof of \olref[fol][seq][sou]{sequent-soundness}.
\end{prob}

\begin{cor}
\ollabel{weak-soundness}
If $\Proves[\Log{LK}] !A$ then $!A$ is valid.
\end{cor}

\begin{cor}
\ollabel{entailment-soundness}
If $\Gamma \Proves[\Log{LK}] !A$ then $\Gamma \Entails !A$.
\end{cor}

\begin{proof}
If $\Gamma \Proves[\Log{LK}] !A$ then for some finite subset $\Gamma_0 
\subseteq \Gamma$, there is !!a{derivation} of $\Gamma_0 \Sequent !A$.  By 
\olref{sequent-soundness}, $\Gamma_0 \Sequent !A$ is valid, i.e. for all $
\Struct M$, $s$  with $\Sat{M}{\Gamma_0}[s]$ we have $\Sat{M}{!A}[s]$. Hence, 
if $\Sat{M}{\Gamma}[s]$, then also $\Sat{M}{!A}[s]$.
\end{proof}

\begin{cor}
\ollabel{consistency-soundness}
If $\Gamma$ is satisfiable, then $\Gamma$ is consistent.
\end{cor}

\begin{proof}
We prove the contrapositive.  Suppose that $\Gamma$ is not consistent, i.e. $
 \Gamma \Proves[\Log{LK}] \lfalse$. By \olref{entailment-soundness}, we get $ 
\Gamma \Entails \lfalse$, i.e. $\Gamma$ is not satisfiable.
\end{proof}

\end{document}
